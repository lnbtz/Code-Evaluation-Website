% Inhalte in Dateien, die mit \include eingefügt werden, fangen immer auf
% einer neuen Seite an.  Dort sollte also beispielsweise wie hier ein neues
% Kapitel (\chapter) anfangen.

\chapter{Allgemeines}

% In diesem kurzen Abschnitt sieht man unter anderem, wie man Fußnoten und
% Links zu Internetquellen einfügt.  Außerdem gibt es zwei Beispiele für das
% Zitieren von Quellen.

Diese Vorlage zur Verwendung mit \LaTeX\footnote{In diesem Dokument wird den
  üblichen Gepflogenheiten entsprechend nicht zwischen dem zugrundeliegenden
  Textsatzsystem \TeX\ und dem weitverbreiteten Makropaket \LaTeX\ für dieses
  System unterschieden, obwohl das rein technisch falsch ist.} kann für die
Formatierung Ihrer Bachelorarbeit verwendet werden, ist jedoch keine
verbindliche Vorgabe; bitte stimmen Sie sich hier mit Ihrer betreuenden
Erstprüferin bzw.\ Ihrem betreuenden Erstprüfer ab.

Den aktuellen Stand dieser Vorlage entnehmen Sie bitte dem Datum auf der
Titelseite.

Das Textsatzsystem \LaTeX\ ist für die Erstellung druckfertiger
wissenschaftlicher Arbeiten unabhängig von deren Umfang hervorragend geeignet,
weil es explizit dafür konzipiert ist.  Im Allgemeinen sollten Sie damit
bessere Ergebnisse als mit Textverarbeitungsprogrammen wie \textsc{Word} oder
\textsc{Pages} erzielen.  Allerdings handelt es sich nicht um ein mit der Maus
zu bedienendes \href{https://de.wikipedia.org/wiki/WYSIWYG}{WYSIWYG}-Programm
und erfordert eine gewisse Einarbeitungszeit.  Die Verwendung von \LaTeX\ für
eine Abschlussarbeit ist daher nicht unbedingt zu empfehlen, wenn Sie das
System erst kurz vor dem Schreiben der Arbeit zum ersten Mal benutzen.

In diesem Sinne ist das vorliegende Dokument auch explizit \textit{nicht} als
Einführung in \LaTeX\ gedacht, sondern setzt voraus, dass Sie schon
ausreichende Kenntnisse mitbringen.  Die können Sie beispielsweise durch die
Lektüre von Büchern wie
\parencite{voss} oder \parencite{schlosser} erwerben.
